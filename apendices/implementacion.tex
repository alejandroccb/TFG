% !TeX root = ../libro.tex
% !TeX encoding = utf8

\chapter{Implementación en SageMath del Algoritmo de Sugiyama}\label{ap:codigo}

En este anexo vamos a documentar las clases desarrolladas para la implementación en SageMath del algoritmo de Sugiyama, tanto en su versión para códigos BCH como para códigos convolucionales  sesgados Reed-Solomon.

En el caso del algoritmo de Sugiyama para códigos BCH se ha implementado un decodificador para códigos BCH, \texttt{BCHSugiyamaDecoder}, que hereda de la clase \texttt{Decoder}. Por otro lado, para la construcción de códigos convolucionales sesgados RS se ha implementado la clase \texttt{SkewRSConvolutionalCode} que hereda de \texttt{AbstractLinearCode}  y utiliza una implementación de los polinomios sesgados de SageMath. Para la codificación de mensajes, se ha creado la clase \texttt{SkewRSConvolutionalEncoder}, que hereda de \texttt{Encoder} y es capaz de proporcionar una palabra código a partir de un mensaje mediante el polinomio generador del código. Por último, se ha implementado un decodificador \texttt{SkewRSSugiyamaDecoder} que hereda de \texttt{Decoder} y utiliza el algoritmo de Sugiyama para decodificar las palabras código recibidas.

El uso de estas clases es sencillo, únicamente se deben cargar los ficheros proporcionados mediante el comando \texttt{load()} de SageMath tal y como se indica a continuación:

\begin{lstlisting}[gobble=4]
    sage: load("bch-sugiyama.sage")
    sage: load("sccc-sugiyama.sage")
\end{lstlisting}

\section{Decodificador para códigos BCH basado en algoritmo de Sugiyama}

En esta sección vamos a describir la clase \texttt{BCHSugiyamaDecoder}, con la cual podemos \emph{decodificar códigos BCH} utilizando el algoritmo de Sugiyama para códigos BCH descrito en \ref{alg:bchsugiyama}.


\begin{description}[leftmargin=1em, font=\normalfont\ttfamily, style=nextline]
    \item[class BCHSugiyamaDecoder(self,code)]
    
    \emph{Hereda de: } \texttt{Decoder}

    Construye un decodificador para códigos BCH utilizando el algoritmo de Sugiyama para códigos BCH.

    \textsc{Argumentos}

    \begin{description}[font=\normalfont\ttfamily]
        \item[code] Código BCH asociado a este decodificador, instancia de \texttt{BCHCode}.
    \end{description}

    \textsc{Ejemplos}

\begin{lstlisting}[gobble=12]
            sage: F = GF(2)
            sage: C = BCHCode(F,15,7,offset = 1)
            sage: D = BCHSugiyamaDecoder(C)
            sage: D
            > Sugiyama decoder for [15, 5] BCH Code over GF(2) with designed distance 7
\end{lstlisting}

\begin{description}[font=\ttfamily,style=nextline]
    \item[correction\_capability(self)] Calcula el número de errores que el código asociado a \texttt{self} puede corregir.
    
    \textsc{Salida}

    \begin{description}[font=\normalfont\ttfamily]
        \item[] Número de errores que el código puede corregir.
    \end{description}
    
    \textsc{Ejemplos}

\begin{lstlisting}[gobble=12]
            sage: F = GF(2)
            sage: C = BCHCode(F,15,7,offset = 1)
            sage: D = BCHSugiyamaDecoder(C)
            sage: D.correction_capability()
            > 3
\end{lstlisting}

\end{description}
    
    \begin{description}[font=\ttfamily,style=nextline]
        \item[decode\_to\_code(self,word)] 
        Corrige los errores de \texttt{word} y devuelve una palabra código asociada al código de \texttt{self}.
        
        \textsc{Argumentos}

        \begin{description}[font=\normalfont\ttfamily]
        \item[word] Polinomio que representa el mensaje recibido que queremos decodificar. 
        \end{description}

        \textsc{Salida}

        \begin{description}[font=\normalfont\ttfamily]
            \item[] Palabra código decodificada, correspondiente a la palabra código corregida.
        \end{description}

        \textsc{Ejemplos}

\begin{lstlisting}[gobble=12]
            sage: F = GF(2)
            sage: C = BCHCode(F,15,7,offset = 1)
            sage: D = BCHSugiyamaDecoder(C)
            sage: message = vector([1,0,1,0,1])
            sage: codeword = message * C.generator_matrix()
            sage: codeword
            > (1, 1, 0, 1, 1, 0, 0, 1, 0, 1, 0, 0, 0, 0, 1)
            sage: R = PolynomialRing(GF(2) ,'x')
            sage: polynomial_codeword = R(list(codeword))
            sage: polynomial_codeword
            > x^14 + x^9 + x^7 + x^4 + x^3 + x + 1
            sage: error = vector([1, 1, 0, 0, 0, 0, 0, 0 , 0, 1, 0, 0, 0, 0, 0])
            sage : received_message = codeword + error
            sage : received_message
            > (0, 0, 0, 1, 1, 0, 0, 1, 0, 0, 0, 0, 0, 0, 1)
            sage: y = R(list(received_message))
            sage: y
            > x^14 + x^7 + x^4 + x^3
            sage: decoded_codeword = D.decode_to_code(y)
            sage: decoded_codeword
            > x^14 + x^9 + x^7 + x^4 + x^3 + x + 1
            sage: decoded_codeword == polynomial_codeword
            > True
\end{lstlisting}
    \end{description}
\end{description}

\section{Clase para códigos convolucionales sesgados RS}

Esta clase implementa un sistema de construcción de códigos convolucionales sesgados Reed-Solomon. Es decir, proporciona métodos para el cálculo del código convolucional cíclico sesgado definido por un polinomio generador perteneciente al anillo de polinomios de Ore sobre el cuerpo de fracciones correspondiente.

\begin{description}[leftmargin=1em, font=\normalfont\ttfamily, style=nextline]
    \item[class SkewRSConvolutionalCode(self,generator\_pol=None,roots=None,inverse\_aut=None)]
    
    \emph{Hereda de: } \texttt{AbstractLinearCode}

    Representación de un código convolucional sesgado Reed-Solomon.

    \textsc{Argumentos}

    \begin{description}[font=\normalfont\ttfamily]
        \item[generator\_pol] (por defecto: \texttt{None}) Polinomio generador del código. 
        \item[roots] (por defecto: \texttt{None}) Una lista de raíces cuyo mínimo común múltiplo por la izquierda conforma el polinomio generador del código.
        \item[inverse\_aut] (por defecto: \texttt{None}) El inverso del automorfismo utilizado para definir la extensión de Ore.
        \item[alpha] (por defecto: \texttt{None}) Un elemento tal que $\{\alpha,\sigma(\alpha),\dots,\sigma^{n-1}(\alpha)\}$ forma una base normal de la extensión de cuerpos $\F(t)^\sigma \subseteq \F(t)$, siendo $n$ es la longitud del código. 
    \end{description}

    \textsc{Ejemplos}

    \begin{lstlisting}[gobble=12]
            sage: F = GF(8, 'a')
            sage: R = PolynomialRing(F,'t')
            sage: K.<t> = FractionField(R)
            sage: aut = K.hom([(t+a)/t])
            sage: inverse_aut = K.hom([a/(t-1)])
            sage: P.<x> = SkewPolynomialRing(K, aut)
            sage: P
            > Ore Polynomial Ring in x over Fraction Field of Univariate Polynomial Ring in t over Finite Field in a of size 2^3 twisted by t |--> (t + a)/t
            sage: alpha = t
            sage: beta = alpha^(-1)*aut(alpha)
            sage: roots = []
            sage: for i in range(4):
            sage:   aut_i = aut^i
            sage:   roots.append(x - aut_i(beta))
            sage: roots 
            > [x + (t + a)/t^2, x + ((a + 1)*t^2 + a*t)/(t^2 + a^2), x + (a*t^2 + (a + 1)*t + a^2 + 1)/(t^2 + a + 1),x + (a*t^2 + (a + 1)*t + a^2)/(t^2 + a)]
            sage: C = SkewRSConvolutionalCode(roots = roots,inverse_aut = inverse_aut, alpha = alpha) 
            sage: C
            > [7,3] Skew Reed-Solomon Convolutional Code on Ore Polynomial Ring in x over Fraction Field of Univariate Polynomial Ring in t over Finite Field in a of size 2^3 twisted by t |--> (t + a)/t with designed distance 5.
    \end{lstlisting}

    \begin{description}[font=\ttfamily,style=nextline]
        \item[generator\_polynomial(self)] 
            Devuelve un polinomio generador del código asociado.

        \textsc{Ejemplos}

        \begin{lstlisting}[gobble=12]
            sage: F = GF(8, 'a')
            sage: R = PolynomialRing(F,'t')
            sage: K.<t> = FractionField(R)
            sage: aut = K.hom([(t+a)/t])
            sage: P.<x> = SkewPolynomialRing(K, aut)
            sage: roots = [x + (t + a)/t^2, x + ((a + 1)*t^2 + a*t)/(t^2 + a^2), x + (a*t^2 + (a + 1)*t + a^2 + 1)/(t^2 + a + 1), x + (a*t^2 + (a + 1)*t + a^2)/(t^2 + a)]
            sage: C = SkewRSConvolutionalCode(roots = roots) 
            sage: C.generator_polynomial()
            > x^4 + (((a^2 + a)*t + a^2)/(t^4 + a^2 + a + 1))*x^3 + ((a^2*t^6 + a*t^5 + a*t^4 + (a^2 + 1)*t^2 + (a^2 + a + 1)*t + a^2 + a + 1)/(t^5 + t^4 + (a^2 + a + 1)*t + a^2 + a + 1))*x^2 + (((a + 1)*t^7 + a^2*t^6 + a^2*t^5 + (a^2 + a + 1)*t^3 + (a^2 + a)*t^2 + (a^2 + a)*t)/(t^6 + a*t^5 + (a + 1)*t^4 + (a^2 + a + 1)*t^2 + (a^2 + 1)*t + a))*x + (a^2*t^6 + (a^2 + a)*t^5 + a^2*t^4 + a^2*t^3 + a*t)/(t^6 + (a^2 + 1)*t^5 + t^4 + (a^2 + a + 1)*t^2 + (a^2 + a)*t + a^2 + a + 1)
        \end{lstlisting}

    \end{description}

    \begin{description}[font=\ttfamily,style=nextline]
        \item[polynomial\_ring(self)] 
        Devuelve el anillo de polinomios sesgados asociado al código.

        \textsc{Ejemplos}

        \begin{lstlisting}[gobble=12]
            sage: F = GF(8, 'a')
            sage: R = PolynomialRing(F,'t')
            sage: K.<t> = FractionField(R)
            sage: aut = K.hom([(t+a)/t])
            sage: P.<x> = SkewPolynomialRing(K, aut)
            sage: roots = [x + (t + a)/t^2, x + ((a + 1)*t^2 + a*t)/(t^2 + a^2), x + (a*t^2 + (a + 1)*t + a^2 + 1)/(t^2 + a + 1), x + (a*t^2 + (a + 1)*t + a^2)/(t^2 + a)]
            sage: C = SkewRSConvolutionalCode(roots = roots) 
            sage: C.polynomial_ring()
            > Ore Polynomial Ring in x over Fraction Field of Univariate Polynomial Ring in t over Finite Field in a of size 2^3 twisted by t |--> (t + a)/t
        \end{lstlisting}        
    \end{description}

    \begin{description}[font=\ttfamily,style=nextline]
        \item[base\_field(self)] 
        Devuelve el cuerpo de fracciones base del código.

        \textsc{Ejemplos}

        \begin{lstlisting}[gobble=12]
            sage: F = GF(8, 'a')
            sage: R = PolynomialRing(F,'t')
            sage: K.<t> = FractionField(R)
            sage: aut = K.hom([(t+a)/t])
            sage: P.<x> = SkewPolynomialRing(K, aut)
            sage: roots = [x + (t + a)/t^2, x + ((a + 1)*t^2 + a*t)/(t^2 + a^2), x + (a*t^2 + (a + 1)*t + a^2 + 1)/(t^2 + a + 1), x + (a*t^2 + (a + 1)*t + a^2)/(t^2 + a)]
            sage: C = SkewRSConvolutionalCode(roots = roots) 
            sage: C.base_field()
            > Fraction Field of Univariate Polynomial Ring in t over Finite Field in a of size 2^3
        \end{lstlisting} 
    \end{description}
    \begin{description}[font=\ttfamily,style=nextline]
        \item[length(self)]
         Devuelve la longitud del código.

         \textsc{Ejemplos}

         \begin{lstlisting}[gobble=12]
             sage: F = GF(8, 'a')
             sage: R = PolynomialRing(F,'t')
             sage: K.<t> = FractionField(R)
             sage: aut = K.hom([(t+a)/t])
             sage: P.<x> = SkewPolynomialRing(K, aut)
             sage: roots = [x + (t + a)/t^2, x + ((a + 1)*t^2 + a*t)/(t^2 + a^2), x + (a*t^2 + (a + 1)*t + a^2 + 1)/(t^2 + a + 1), x + (a*t^2 + (a + 1)*t + a^2)/(t^2 + a)]
             sage: C = SkewRSConvolutionalCode(roots = roots) 
             sage: C.length()
             > 7
         \end{lstlisting} 
    \end{description}
    \begin{description}[font=\ttfamily,style=nextline]
        \item[dimension(self)] 
        Devuelve la dimensión del código.
    
        \textsc{Ejemplos}

        \begin{lstlisting}[gobble=12]
            sage: F = GF(8, 'a')
            sage: R = PolynomialRing(F,'t')
            sage: K.<t> = FractionField(R)
            sage: aut = K.hom([(t+a)/t])
            sage: P.<x> = SkewPolynomialRing(K, aut)
            sage: roots = [x + (t + a)/t^2, x + ((a + 1)*t^2 + a*t)/(t^2 + a^2), x + (a*t^2 + (a + 1)*t + a^2 + 1)/(t^2 + a + 1), x + (a*t^2 + (a + 1)*t + a^2)/(t^2 + a)]
            sage: C = SkewRSConvolutionalCode(roots = roots) 
            sage: C.dimension()
            > 3
        \end{lstlisting} 
    \end{description}

    \begin{description}[font=\ttfamily,style=nextline]
        \item[designed\_distance(self)] 
        Devuelve la distancia mínima prevista del código
    
        \textsc{Ejemplos}

        \begin{lstlisting}[gobble=12]
            sage: F = GF(8, 'a')
            sage: R = PolynomialRing(F,'t')
            sage: K.<t> = FractionField(R)
            sage: aut = K.hom([(t+a)/t])
            sage: P.<x> = SkewPolynomialRing(K, aut)
            sage: roots = [x + (t + a)/t^2, x + ((a + 1)*t^2 + a*t)/(t^2 + a^2), x + (a*t^2 + (a + 1)*t + a^2 + 1)/(t^2 + a + 1), x + (a*t^2 + (a + 1)*t + a^2)/(t^2 + a)]
            sage: C = SkewRSConvolutionalCode(roots = roots) 
            sage: C.designed_distance()
            > 5
        \end{lstlisting} 
    \end{description}

    \begin{description}[font=\ttfamily,style=nextline]
        \item[automorphism(self)] 
        Devuelve el automorfismo asociado al anillo de polinomios sesgados.
    
        \textsc{Ejemplos}

        \begin{lstlisting}[gobble=12]
            sage: F = GF(8, 'a')
            sage: R = PolynomialRing(F,'t')
            sage: K.<t> = FractionField(R)
            sage: aut = K.hom([(t+a)/t])
            sage: P.<x> = SkewPolynomialRing(K, aut)
            sage: roots = [x + (t + a)/t^2, x + ((a + 1)*t^2 + a*t)/(t^2 + a^2), x + (a*t^2 + (a + 1)*t + a^2 + 1)/(t^2 + a + 1), x + (a*t^2 + (a + 1)*t + a^2)/(t^2 + a)]
            sage: C = SkewRSConvolutionalCode(roots = roots) 
            sage: C.automorphism()
            > Ring endomorphism of Fraction Field of Univariate Polynomial Ring in t over Finite Field in a of size 2^3
            Defn: t |--> (t + a)/t
        \end{lstlisting} 
    \end{description}

    \begin{description}[font=\ttfamily,style=nextline]
        \item[inverse\_automorphism(self)] 
        Devuelve el inverso del automorfismo asociado al anillo de polinomios sesgados.

        \textsc{Ejemplos}

        \begin{lstlisting}[gobble=12]
            sage: F = GF(8, 'a')
            sage: R = PolynomialRing(F,'t')
            sage: K.<t> = FractionField(R)
            sage: aut = K.hom([(t+a)/t])
            sage: inverse_aut = a/(t-1)
            sage: P.<x> = SkewPolynomialRing(K, aut)
            sage: roots = [x + (t + a)/t^2, x + ((a + 1)*t^2 + a*t)/(t^2 + a^2), x + (a*t^2 + (a + 1)*t + a^2 + 1)/(t^2 + a + 1), x + (a*t^2 + (a + 1)*t + a^2)/(t^2 + a)]
            sage: C = SkewRSConvolutionalCode(roots = roots,inverse_aut = inverse_aut) 
            sage: C.inverse_automorphism()
            > Ring endomorphism of Fraction Field of Univariate Polynomial Ring in t over Finite Field in a of size 2^3
            Defn: t |--> a/(t + 1)
        \end{lstlisting} 
    \end{description}
\end{description}

\section{Codificador de códigos convolucionales sesgados RS}

En esta sección presentaremos la implementación de un codificador para los códigos convolucionales sesgados RS. Esta clase nos proporcionará un método para codificar un mensaje.

\begin{description}[leftmargin=1em, font=\normalfont\ttfamily, style=nextline]
    \item[class SkewRSConvolutionalEncoder(self,code)]
    
    \emph{Hereda de: } \texttt{Encoder}

    Representación de un codificador para un código convolucional sesgado Reed-Solomon mediante su polinomio generador.

    \textsc{Argumentos}

    \begin{description}[leftmargin=1em, font=\normalfont\ttfamily]
    \item[code] Código asociado a este codificador.
    \end{description}

    \textsc{Ejemplos}

    \begin{lstlisting}[gobble=7]
        sage: F = GF(8, 'a')
        sage: R = PolynomialRing(F,'t')
        sage: K.<t> = FractionField(R)
        sage: aut = K.hom([(t+a)/t])
        sage: P.<x> = SkewPolynomialRing(K, aut)
        sage: roots = [x + (t + a)/t^2, x + ((a + 1)*t^2 + a*t)/(t^2 + a^2), x + (a*t^2 + (a + 1)*t + a^2 + 1)/(t^2 + a + 1), x + (a*t^2 + (a + 1)*t + a^2)/(t^2 + a)]
        sage: C = SkewRSConvolutionalCode(roots = roots) 
        sage: SkewRSConvolutionalEncoder(C)
        > SkewRSConvolutionalEncoder for a [7,3] Skew Reed-Solomon Convolutional Code on Ore Polynomial Ring in x over Fraction Field of Univariate Polynomial Ring in t over Finite Field in a of size 2^3 twisted by t |--> (t + a)/t with designed distance 5.
    \end{lstlisting}

    \begin{description}[font=\ttfamily,style=nextline]
        \item[encode(self,message\_poly)] 
        Codifica un mensaje dado en forma polinomial.

        \textsc{Argumentos}

        \begin{description}[font=\normalfont\ttfamily]
            \item[message\_poly]: Mensaje en forma polinomial. 
        \end{description}

        \textsc{Salida}

        \begin{description}[font=\normalfont\ttfamily]
            \item[] Palabra código resultante de la codificación del mensaje.
        \end{description}

        \textsc{Ejemplos}

        \begin{lstlisting}[gobble=10]
            sage: F = GF(8, 'a')
            sage: a = F.gen()
            sage: R = PolynomialRing(F,'t')
            sage: K.<t> = FractionField(R)
            sage: aut = K.hom([(t+a)/t])
            sage: P.<x> = SkewPolynomialRing(K, aut)
            sage: roots = [x + (t + a)/t^2, x + ((a + 1)*t^2 + a*t)/(t^2 + a^2), x + (a*t^2 + (a + 1)*t + a^2 + 1)/(t^2 + a + 1), x + (a*t^2 + (a + 1)*t + a^2)/(t^2 + a)]
            sage: C = SkewRSConvolutionalCode(roots = roots) 
            sage: E = SkewRSConvolutionalEncoder(C)
            sage: E.encode(x + a)
            > x^5 + ((a^2*t^4 + a*t^3 + a)/(t^4 + 1))*x^4 + ((a^2*t^10 + a^2*t^9 + (a^2 + a)*t^8 + a^2*t^6 + (a^2 + a + 1)*t^4 + a*t^2 + (a^2 + a)*t + a)/(t^9 + (a^2 + a)*t^5 + (a^2 + a + 1)*t))*x^3 + ((t^10 + (a^2 + 1)*t^9 + (a^2 + 1)*t^8 + a*t^7 + t^6 + t^5 + a^2*t^3 + t^2 + (a^2 + a)*t + a^2)/(t^10 + t^9 + (a^2 + a)*t^6 + (a^2 + a)*t^5 + (a^2 + a + 1)*t^2 + (a^2 + a + 1)*t))*x^2 + (((a^2 + a)*t^11 + (a + 1)*t^10 + (a + 1)*t^9 + (a^2 + a + 1)*t^7 + (a + 1)*t^6 + t^5 + (a^2 + a + 1)*t^4 + a^2*t^3 + a^2*t^2 + t + a + 1)/(t^10 + a*t^9 + (a + 1)*t^8 + (a^2 + a)*t^6 + (a^2 + a + 1)*t^5 + t^4 + (a^2 + a + 1)*t^2 + (a^2 + 1)*t + a))*x + ((a + 1)*t^6 + (a^2 + a + 1)*t^5 + (a + 1)*t^4 + (a + 1)*t^3 + a^2*t)/(t^6 + (a^2 + 1)*t^5 + t^4 + (a^2 + a + 1)*t^2 + (a^2 + a)*t + a^2 + a + 1)
        \end{lstlisting}        
    \end{description}

    \begin{description}[font=\ttfamily,style=nextline]
        \item[unencode(self,codeword\_poly)] 
        Devuelve el mensaje correspondiente a \texttt{codeword\_poly}.

        \textsc{Argumentos}

        \begin{description}[font=\normalfont\ttfamily]
            \item[codeword\_poly]: Palabra código en forma polinomial. 
        \end{description}

        \textsc{Salida}

        \begin{description}[font=\normalfont\ttfamily]
            \item[] Mensaje correspondiente a la palabra código.
        \end{description}

        \textsc{Ejemplos}

        \begin{lstlisting}[gobble=10]
            sage: F = GF(8, 'a')
            sage: a = F.gen()
            sage: R = PolynomialRing(F,'t')
            sage: K.<t> = FractionField(R)
            sage: aut = K.hom([(t+a)/t])
            sage: P.<x> = SkewPolynomialRing(K, aut)
            sage: roots = [x + (t + a)/t^2, x + ((a + 1)*t^2 + a*t)/(t^2 + a^2), x + (a*t^2 + (a + 1)*t + a^2 + 1)/(t^2 + a + 1), x + (a*t^2 + (a + 1)*t + a^2)/(t^2 + a)]
            sage: C = SkewRSConvolutionalCode(roots = roots) 
            sage: E = SkewRSConvolutionalEncoder(C)
            sage: codeword = x^5 + ((a^2*t^4 + a*t^3 + a)/(t^4 + 1))*x^4 + ((a^2*t^10 + a^2*t^9 + (a^2 + a)*t^8 + a^2*t^6 + (a^2 + a + 1)*t^4 + a*t^2 + (a^2 + a)*t + a)/(t^9 + (a^2 + a)*t^5 + (a^2 + a + 1)*t))*x^3 + ((t^10 + (a^2 + 1)*t^9 + (a^2 + 1)*t^8 + a*t^7 + t^6 + t^5 + a^2*t^3 + t^2 + (a^2 + a)*t + a^2)/(t^10 + t^9 + (a^2 + a)*t^6 + (a^2 + a)*t^5 + (a^2 + a + 1)*t^2 + (a^2 + a + 1)*t))*x^2 + (((a^2 + a)*t^11 + (a + 1)*t^10 + (a + 1)*t^9 + (a^2 + a + 1)*t^7 + (a + 1)*t^6 + t^5 + (a^2 + a + 1)*t^4 + a^2*t^3 + a^2*t^2 + t + a + 1)/(t^10 + a*t^9 + (a + 1)*t^8 + (a^2 + a)*t^6 + (a^2 + a + 1)*t^5 + t^4 + (a^2 + a + 1)*t^2 + (a^2 + 1)*t + a))*x + ((a + 1)*t^6 + (a^2 + a + 1)*t^5 + (a + 1)*t^4 + (a + 1)*t^3 + a^2*t)/(t^6 + (a^2 + 1)*t^5 + t^4 + (a^2 + a + 1)*t^2 + (a^2 + a)*t + a^2 + a + 1)
            sage: E.unencode(codeword)
            > x + a
        \end{lstlisting}        
    \end{description}
\end{description}

\section{Decodificador para códigos convolucionales cíclicos sesgados RS basado en algoritmo de Sugiyama}

En esta sección se va a explicar la implementación en SageMath del decodificador para códigos convolucionales sesgados RS mediante el algoritmo de Sugiyama expuesto en \ref{ch:quinto-capitulo}.

\begin{description}[leftmargin=1em, font=\normalfont\ttfamily, style=nextline]
    \item[class SkewRSSugiyamaDecoder(self,code)]
    
    \emph{Hereda de: } \texttt{Decoder}

    Construye un decodificador para códigos convolucionales de Reed-Solomon sesgados basado en el algoritmo de Sugiyama para códigos convolucionales sesgados RS.

    \textsc{Argumentos}

    \begin{description}[font=\normalfont\ttfamily]
        \item[code] Código asociado al decodificador.
    \end{description}

    \textsc{Ejemplos}

    \begin{lstlisting}[gobble=8]
        sage: F = GF(8, 'a')
        sage: a = F.gen()
        sage: R = PolynomialRing(F,'t')
        sage: K.<t> = FractionField(R)
        sage: aut = K.hom([(t+a)/t])
        sage: P.<x> = SkewPolynomialRing(K, aut)
        sage: roots = [x + (t + a)/t^2, x + ((a + 1)*t^2 + a*t)/(t^2 + a^2), x + (a*t^2 + (a + 1)*t + a^2 + 1)/(t^2 + a + 1), x + (a*t^2 + (a + 1)*t + a^2)/(t^2 + a)]
        sage: C = SkewRSConvolutionalCode(roots = roots) 
        sage: SkewRSSugiyamaDecoder(C)
        > <A Sugiyama-like algorithm for [7,3] Skew Reed-Solomon Convolutional Code on Ore Polynomial Ring in x over Fraction Field of Univariate Polynomial Ring in t over Finite Field in a of size 2^3 twisted by t |--> (t + a)/t with designed distance 5 with correction capability of 2 errors>
    \end{lstlisting}     


    \begin{description}[font=\ttfamily,style = nextline]
        \item[correction\_capability(self)] Devuelve la capacidad de corrección de errores del código.    
    \end{description}

    \textsc{Ejemplos}

    \begin{lstlisting}[gobble=8]
        sage: F = GF(8, 'a')
        sage: a = F.gen()
        sage: R = PolynomialRing(F,'t')
        sage: K.<t> = FractionField(R)
        sage: aut = K.hom([(t+a)/t])
        sage: P.<x> = SkewPolynomialRing(K, aut)
        sage: roots = [x + (t + a)/t^2, x + ((a + 1)*t^2 + a*t)/(t^2 + a^2), x + (a*t^2 + (a + 1)*t + a^2 + 1)/(t^2 + a + 1), x + (a*t^2 + (a + 1)*t + a^2)/(t^2 + a)]
        sage: C = SkewRSConvolutionalCode(roots = roots) 
        sage: D = SkewRSSugiyamaDecoder(C)
        sage: D.correction_capability()
        > 2
    \end{lstlisting}     

    \begin{description}[font=\ttfamily,style = nextline]
        \item[decode\_to\_code(self,word)] 
        Corrige los errores de \texttt{word} y devuelve una palabra código del código asociado a \texttt{self}.
        
        \textsc{Argumentos}

        \begin{description}[font=\normalfont\ttfamily]
            \item[word] Polinomio que representa el mensaje recibido que queremos decodificar.
        \end{description}

        \textsc{Ejemplos}


        \begin{lstlisting}[gobble=8]
            sage: F = GF(8, 'a')
            sage: a = F.gen()
            sage: R = PolynomialRing(F,'t')
            sage: K.<t> = FractionField(R)
            sage: aut = K.hom([(t+a)/t])
            sage: P.<x> = SkewPolynomialRing(K, aut)
            sage: roots = [x + (t + a)/t^2, x + ((a + 1)*t^2 + a*t)/(t^2 + a^2), x + (a*t^2 + (a + 1)*t + a^2 + 1)/(t^2 + a + 1), x + (a*t^2 + (a + 1)*t + a^2)/(t^2 + a)]
            sage: C = SkewRSConvolutionalCode(roots = roots) 
            sage: D = SkewRSSugiyamaDecoder(C)
            sage: g = C.generator_polynomial()
            sage: error = ((t + 2*a)/(1+t))*x^3 + (t/(2*t + 1))*x
            sage: received_word = g - error
            sage: decoded_word = D.decode_to_code(received_word)
            sage: decoded_word
            > x^4 + (((a^2 + a)*t + a^2)/(t^4 + a^2 + a + 1))*x^3 + ((a^2*t^6 + a*t^5 + a*t^4 + (a^2 + 1)*t^2 + (a^2 + a + 1)*t + a^2 + a + 1)/(t^5 + t^4 + (a^2 + a + 1)*t + a^2 + a + 1))*x^2 + (((a + 1)*t^7 + a^2*t^6 + a^2*t^5 + (a^2 + a + 1)*t^3 + (a^2 + a)*t^2 + (a^2 + a)*t)/(t^6 + a*t^5 + (a + 1)*t^4 + (a^2 + a + 1)*t^2 + (a^2 + 1)*t + a))*x + (a^2*t^6 + (a^2 + a)*t^5 + a^2*t^4 + a^2*t^3 + a*t)/(t^6 + (a^2 + 1)*t^5 + t^4 + (a^2 + a + 1)*t^2 + (a^2 + a)*t + a^2 + a + 1)
            sage: decoded_word == g
            > True
        \end{lstlisting}     
    \end{description}
\end{description}


\section{Funciones auxiliares}

Para la implementación de las clases anteriores, fue necesario el uso de funciones auxiliares. En esta sección, vamos a dar una breve descripción de ellas.



\begin{description}[leftmargin=1em, font=\ttfamily, style=nextline]
    \item[left\_lcm(poly\_list)]
    
    Calcula el mínimo común múltiplo por la izquierda de una lista de polinomios sesgados.

    \textsc{Argumentos}

    \begin{description}[font=\ttfamily]
        \item[poly\_list] Lista de polinomios para los cuales se calculará el LCM por la izquierda.
    \end{description}

    \textsc{Salida}

    \begin{description}[font=\normalfont\ttfamily]
        \item[] El MCM por la izquierda de la lista de polinomios dada.
    \end{description}

    \textsc{Ejemplos}

    \begin{lstlisting}[gobble=8]
        sage: F = GF(8, 'a')
        sage: a = F.gen()
        sage: R = PolynomialRing(F,'t')
        sage: K.<t> = FractionField(R)
        sage: aut = K.hom([(t+a)/t])
        sage: P.<x> = SkewPolynomialRing(K, aut)
        sage: roots = [x + (t + a)/t^2, x + ((a + 1)*t^2 + a*t)/(t^2 + a^2), x + (a*t^2 + (a + 1)*t + a^2 + 1)/(t^2 + a + 1), x + (a*t^2 + (a + 1)*t + a^2)/(t^2 + a)]
        sage: left_lcm(roots)
        > x^4 + (((a^2 + a)*t + a^2)/(t^4 + a^2 + a + 1))*x^3 + ((a^2*t^6 + a*t^5 + a*t^4 + (a^2 + 1)*t^2 + (a^2 + a + 1)*t + a^2 + a + 1)/(t^5 + t^4 + (a^2 + a + 1)*t + a^2 + a + 1))*x^2 + (((a + 1)*t^7 + a^2*t^6 + a^2*t^5 + (a^2 + a + 1)*t^3 + (a^2 + a)*t^2 + (a^2 + a)*t)/(t^6 + a*t^5 + (a + 1)*t^4 + (a^2 + a + 1)*t^2 + (a^2 + 1)*t + a))*x + (a^2*t^6 + (a^2 + a)*t^5 + a^2*t^4 + a^2*t^3 + a*t)/(t^6 + (a^2 + 1)*t^5 + t^4 + (a^2 + a + 1)*t^2 + (a^2 + a)*t + a^2 + a + 1)
    \end{lstlisting}     

\end{description}

\begin{description}[leftmargin=1em, font=\ttfamily, style=nextline]
     
    \item[order\_of\_automorphism(K, sigma)]
    
    Calcula el orden del automorfismo \texttt{sigma} en el cuerpo \texttt{K}.

    \textsc{Argumentos}

    \begin{description}[font=\normalfont\ttfamily]
        \item[K] El cuerpo sobre el cual está definido el automorfismo.
        \item[sigma] El automorfismo cuyo orden se va a calcular.
    \end{description}

    \textsc{Salida}

    \begin{description}[font=\normalfont\ttfamily]
        \item[] El orden del automorfismo \texttt{sigma}.
    \end{description}

    \textsc{Ejemplos}

    \begin{lstlisting}[gobble=8]
        sage: F = GF(8, 'a')
        sage: a = F.gen()
        sage: R = PolynomialRing(F,'t')
        sage: K.<t> = FractionField(R)
        sage: aut = K.hom([(t+a)/t])
        sage: order_of_automorphism(K,aut)
        > 7
    \end{lstlisting}     

\end{description}

\begin{description}[leftmargin=1em, font=\ttfamily, style=nextline]

    \item[jth\_norm(gamma, sigma, j)]
    
    Calcula la j-ésima norma del elemento \texttt{gamma} usando el automorfismo \texttt{sigma}.

    \textsc{Argumentos}

    \begin{description}[font=\normalfont\ttfamily]
        \item[gamma] El elemento para el cual se va a calcular la norma.
        \item[sigma] El automorfismo utilizado en el cálculo de la norma.
        \item[j] El orden de la norma.
    \end{description}

    \textsc{Salida}

    \begin{description}[font=\normalfont\ttfamily]
        \item[] La j-ésima norma de \texttt{gamma}.
    \end{description}

    \textsc{Ejemplos}

    \begin{lstlisting}[gobble=8]
        sage: F = GF(8, 'a')
        sage: a = F.gen()
        sage: R = PolynomialRing(F,'t')
        sage: K.<t> = FractionField(R)
        sage: aut = K.hom([(t+a)/t])
        sage: alpha = t
        sage: jth_norm(alpha, aut, 10)
        > (a + 1)*t + a
    \end{lstlisting}     
\end{description}

\begin{description}[leftmargin=1em, font=\ttfamily, style=nextline]
    \item[normal\_basis(aut, c, F)]
        
    Determina si la matriz construida utilizando el automorfismo \texttt{aut} y el elemento \texttt{c} tiene determinante nulo. Esto es equivalente a decir que $\{c,\sigma(c),\dots,\sigma^{n-1}(c)\}$ forma una base normal de la extensión $\F(t)^\sigma \subseteq \F(t)$.
    
    \textsc{Argumentos}
    
    \begin{description}[font=\normalfont\ttfamily]
        \item[F] Cuerpo de fracciones empleado.
        \item[automorphism] Un automorfismo del cuerpo de fracciones \texttt{F}.
        \item[c] Un elemento del cuerpo \texttt{F}.
    \end{description}
    
    \textsc{Salida}
    
    \begin{description}[font=\normalfont\ttfamily]
        \item[] \texttt{bool}: Verdadero si el determinante es distinto de cero, Falso si es cero.
    \end{description}

    \textsc{Ejemplos}

    \begin{lstlisting}[gobble=8]
        sage: F = GF(8, 'a')
        sage: a = F.gen()
        sage: R = PolynomialRing(F,'t')
        sage: K.<t> = FractionField(R)
        sage: aut = K.hom([(t+a)/t])
        sage: c = t
        sage: normal_basis(aut,c,K)
        > True
    \end{lstlisting}  


\end{description} 

\begin{description}[leftmargin=1em, font=\ttfamily, style=nextline]

    \item[RightED(f, g, inverse\_aut)]
    
    Realiza la división euclídea por la derecha en el anillo de polinomios sesgados definido por $\F(t)[x;\sigma]$.

    \textsc{Argumentos}

    \begin{description}[font=\normalfont\ttfamily]
        \item[f] El polinomio sesgado dividendo.
        \item[g] El polinomio sesgado divisor.
        \item[inverse\_aut] El inverso del automorfismo con el que se define la extensión de Ore.
    \end{description}

    \textsc{Salida}

    \begin{description}[font=\normalfont\ttfamily]
        \item[] \texttt{c}: El polinomio cociente resultante de la división de \texttt{f} por \texttt{g}.
        \item[] \texttt{r}: El polinomio resto resultante de la división de \texttt{f} por \texttt{g}.
    \end{description}

    \textsc{Ejemplos}

    \begin{lstlisting}[gobble=8]
        sage: F = GF(8, 'a')
        sage: a = F.gen()
        sage: R = PolynomialRing(F,'t')
        sage: K.<t> = FractionField(R)
        sage: aut = K.hom([(t+a)/t])
        sage: inverse_aut = K.hom([a/(t-1)])
        sage: P.<x> = SkewPolynomialRing(K, aut)
        sage: f = (t^4*a + t^2*a^2 + a)/(t^5 + a)*x^4 + x^2 + 1
        sage: g = (t^4*a + t^2*a^3 + t)/(t^3 + t*a)*x^2 + x + t
        sage: quotient,reminder = RightED(f,g,inverse_aut)
        sage: quotient
        > ((a*t^8 + a^2*t^6 + (a^2 + 1)*t^4 + a^2 + 1)/(t^8 + t^7 + (a^2 + a)*t^6 + t^5 + (a^2 + a)*t^4 + t^3 + t^2 + a^2*t + a^2 + a))*x^2 + (((a^2 + 1)*t^9 + (a^2 + a)*t^8 + t^7 + (a^2 + a)*t^6 + t^5 + (a^2 + 1)*t^4 + a*t^2)/(t^9 + a*t^8 + a*t^7 + t^6 + (a^2 + 1)*t^5 + (a + 1)*t^4 + (a^2 + a)*t^3 + (a + 1)*t^2 + (a^2 + 1)*t + a + 1))*x + ((a^2 + a)*t^12 + a*t^11 + (a^2 + a)*t^10 + (a + 1)*t^9 + (a^2 + a)*t^8 + a^2*t^7 + t^6 + (a^2 + a)*t^5 + (a^2 + a)*t^4 + a^2*t^3 + (a^2 + a + 1)*t^2 + a*t + a^2 + a + 1)/(t^12 + (a^2 + a)*t^11 + t^10 + (a^2 + a)*t^9 + a^2*t^8 + (a^2 + a + 1)*t^7 + t^6 + a^2*t^5 + t^4 + a^2*t^3 + t^2 + (a^2 + a + 1)*t)
        sage: reminder
        > (((a^2 + 1)*t^14 + a*t^13 + t^11 + (a^2 + 1)*t^10 + a^2*t^8 + (a^2 + a)*t^7 + a*t^6 + t^5 + a*t^4 + (a + 1)*t^3 + (a + 1)*t^2 + (a^2 + 1)*t + a)/(t^13 + (a^2 + a + 1)*t^12 + (a^2 + a + 1)*t^11 + (a^2 + a + 1)*t^9 + a*t^8 + (a^2 + a + 1)*t^7 + t^6 + a^2*t^4 + a^2*t^3 + (a^2 + 1)*t^2 + a^2*t + a^2 + a))*x + ((a^2 + a)*t^12 + (a + 1)*t^11 + a*t^9 + (a^2 + a)*t^6 + (a^2 + a + 1)*t^5 + a*t^4 + (a^2 + 1)*t^3 + (a + 1)*t^2 + (a + 1)*t)/(t^11 + (a^2 + a)*t^10 + t^9 + (a^2 + a)*t^8 + a^2*t^7 + (a^2 + a + 1)*t^6 + t^5 + a^2*t^4 + t^3 + a^2*t^2 + t + a^2 + a + 1)
        sage: f == g*quotient + reminder
        > True
    \end{lstlisting} 
\end{description}   

\begin{description}[leftmargin=1em, font=\ttfamily, style=nextline]

    \item[LeftED(f, g, aut)]
    
    Realiza la división euclídea por la izquierda en el anillo de polinomios sesgados definido por $\F(t)[x;\sigma]$.

    \textsc{Argumentos}

    \begin{description}[font=\normalfont\ttfamily]
        \item[f] El polinomio sesgado dividendo.
        \item[g] El polinomio sesgado divisor.
        \item[aut] El automorfismo utilizado para la definición del anillo de polinomios sesgados.
    \end{description}

    \textsc{Salida}

    \begin{description}[font=\normalfont\ttfamily]
        \item[] \texttt{c}: El polinomio cociente resultante de la división de \texttt{f} por \texttt{g}.
        \item[] \texttt{r}: El polinomio resto resultante de la división de \texttt{f} por \texttt{g}.
    \end{description}

    \textsc{Ejemplos}

    \begin{lstlisting}[gobble=8]
        sage: F = GF(8, 'a')
        sage: a = F.gen()
        sage: R = PolynomialRing(F,'t')
        sage: K.<t> = FractionField(R)
        sage: aut = K.hom([(t+a)/t])
        sage: P.<x> = SkewPolynomialRing(K, aut)
        sage: f = (t^4*a + t^2*a^2 + a)/(t^5 + a)*x^4 + x^2 + 1
        sage: g = (t^4*a + t^2*a^3 + t)/(t^3 + t*a)*x^2 + x + t
        sage: quotient,reminder = LeftED(f,g,aut)
        sage: quotient
        > ((a*t^7 + a^2*t^6 + (a^2 + a)*t^5 + (a^2 + a + 1)*t^4 + (a^2 + a)*t^3 + (a^2 + a + 1)*t^2 + a*t + a^2)/(t^8 + (a^2 + 1)*t^7 + (a^2 + a + 1)*t^6 + a*t^3 + t^2 + (a^2 + 1)*t))*x^2 + (((a + 1)*t^8 + (a + 1)*t^7 + a^2*t^6 + (a^2 + 1)*t^5 + (a^2 + a)*t^4 + (a^2 + 1)*t^3 + (a + 1)*t + 1)/(t^8 + a*t^7 + (a^2 + 1)*t^5 + (a^2 + a)*t^4 + (a + 1)*t^3 + (a^2 + a + 1)*t^2 + (a^2 + 1)*t + a^2 + 1))*x + (t^12 + (a^2 + 1)*t^11 + t^10 + a^2*t^9 + a^2*t^8 + (a^2 + a)*t^7 + a*t^6 + (a^2 + 1)*t^5 + a^2*t^4 + a^2 + a + 1)/(t^13 + t^12 + a*t^11 + a^2*t^10 + a*t^9 + (a + 1)*t^8 + (a^2 + a)*t^7 + (a^2 + 1)*t^6 + (a + 1)*t^5 + a^2*t^4 + a*t^3 + (a + 1)*t^2 + a*t)
        sage: reminder
        > (((a + 1)*t^13 + t^12 + a^2*t^11 + (a^2 + 1)*t^10 + (a^2 + a + 1)*t^9 + (a^2 + a + 1)*t^8 + a^2*t^7 + t^6 + (a^2 + a)*t^5 + a*t^4 + (a^2 + 1)*t^3 + t^2 + (a^2 + 1)*t + a^2)/(t^13 + t^12 + a*t^11 + a^2*t^10 + a*t^9 + (a + 1)*t^8 + (a^2 + a)*t^7 + (a^2 + 1)*t^6 + (a + 1)*t^5 + a^2*t^4 + a*t^3 + (a + 1)*t^2 + a*t))*x + (a^2*t^11 + (a + 1)*t^10 + (a^2 + a)*t^8 + (a^2 + 1)*t^7 + a^2*t^6 + (a^2 + a + 1)*t^4 + a^2*t^3 + a*t^2 + (a + 1)*t + a^2 + 1)/(t^12 + t^11 + a*t^10 + a^2*t^9 + a*t^8 + (a + 1)*t^7 + (a^2 + a)*t^6 + (a^2 + 1)*t^5 + (a + 1)*t^4 + a^2*t^3 + a*t^2 + (a + 1)*t + a)
        sage: f == quotient*g + reminder
        > True
    \end{lstlisting} 
\end{description}

\begin{description}[leftmargin=1em, font=\ttfamily, style=nextline]

    \item[REEA(f, g, inverse\_sigma)]
    
    Realiza el Algoritmo de Euclides extendido por la derecha (REEA) en el anillo de polinomios sesgados definido por $\F(t)[x;\sigma]$.

    \textsc{Argumentos}

    \begin{description}[font=\normalfont\ttfamily]
        \item[f] El primer polinomio sesgado.
        \item[g] El segundo polinomio sesgado.
        \item[inverse\_sigma] El inverso del automorfismo utilizado para definir la extensión de Ore.
    \end{description}

    \textsc{Salida}

    \begin{description}[font=\normalfont\ttfamily]
        \item[] \texttt{u}: Lista de coeficientes para la identidad de Bézout tal que \texttt{f * u[i] + g * v[i] = r[i]}.
        \item[]  \texttt{v}: Lista de coeficientes para la identidad de Bézout tal que \texttt{u[i] * f + v[i] * g = r[i]}.
        \item[] \texttt{r}: Lista de restos.
    \end{description}

    \textsc{Ejemplos}

    \begin{lstlisting}[gobble=8]
        sage: F = GF(8, 'a')
        sage: a = F.gen()
        sage: R = PolynomialRing(F,'t')
        sage: K.<t> = FractionField(R)
        sage: aut = K.hom([(t+a)/t])
        sage: inverse_aut = K.hom([a/(t-1)])
        sage: P.<x> = SkewPolynomialRing(K, aut)
        sage: f = (t^4*a + t^2*a^2 + a)/(t^5 + a)*x^4 + x^2 + 1
        sage: g = (t^4*a + t^2*a^3 + t)/(t^3 + t*a)*x^2 + x + t
        sage: u,v,r = REEA(f,g,inverse_aut)
        sage: f*u[2] + g*v[2] == r[2] 
        > True
    \end{lstlisting}  

\end{description}


\section{Desarrollo de la implementación}

Tanto el código desarrollado en este trabajo, como todos los ejemplos que se han expuesto previamente, pueden encontrarse en:

\begin{center}
    \href{https://github.com/alejandroccb/TFG/tree/main/src}{https://github.com/alejandroccb/TFG/tree/main/src}
\end{center}

\subsection{Organización del código}

El árbol de directorios de todo el código y las pruebas realizadas es el siguiente:
    \begin{itemize}
        \item[] \texttt{jupyter}
        \begin{itemize}
            \item[] \texttt{ejemplos\_codigo.ipynb}: Cuaderno de Jupyter donde se encuentran los ejemplos realizados durante la descripción del código.
            \item[] \texttt{ejemplo\_sugiyama.ipynb}: Cuaderno de Jupyter donde se explica paso por paso el Ejemplo \ref{ej:sugiyama}.
        \end{itemize}
        \item[] \texttt{src}
        \begin{itemize}
            \item[] \texttt{bch\_sugiyama.sage}: Contiene la clase \texttt{BCHSugiyamaDecoder} para la decodificación de códigos BCH con el algoritmo de Sugiyama.
            \item[] \texttt{sccc\_sugiyama.sage}: Contiene las clases \texttt{SkewRSConvolutionalCode}: para la construcción de códigos convolucionales sesgados RS, la clase \texttt{SkewRSConvolutionalEncoder} para su codificación y por último la clase \texttt{SkewRSConvolutionalDecoder} para la decodificación con el algoritmo de Sugiyama.
            \item[] \texttt{tests}
            \begin{itemize}
                \item[] \texttt{test\_bch\_sugiyama.py}: Script de Pytest para pruebas.
                \item[] \texttt{test\_sccc\_sugiyama.py}: Script de Pytest para pruebas.
                \item[] \texttt{bch\_sugiyama.py}: Traducción de \texttt{bch\_sugiyama.sage} en Python.
                \item[] \texttt{sccc\_sugiyama.py}: Traducción de \texttt{sccc\_sugiyama.sage} en Python.
            \end{itemize}
        \end{itemize}
    \end{itemize}

\subsection{Tests}
Se han desarrollado distintos tests utilizando la herramienta Pytest \cite{pytest7.4.0}. Estos tests se pueden encontrar en los siguientes ficheros:

\begin{itemize}
    \item  \texttt{test\_bch\_sugiyama.py}
    \item  \texttt{test\_sccc\_sugiyama.py}
\end{itemize}

Para utilizar Pytest es necesario traducir los ficheros de Sage a Python, por eso, también han de incluirse las traducciones en el mismo fichero.

La función principal de estos test es verificar mediante un gran número de pruebas, que las clases implementadas funcionan como se espera. Estos ficheros tienen predefinidos unos cuantos tipos de códigos. El funcionamiento consiste en generar varios mensajes aleatorios para cada código, cuyo tamaño viene determinado por la dimensión del código. Después, los mensajes se codifican y se calcula un vector de errores aleatorio que simula los fallos que se han producido durante la transmisión de las palabras código. Por último, la palabra código con errores se decodifica y se comprueba que coincide con la palabra código original.



\endinput