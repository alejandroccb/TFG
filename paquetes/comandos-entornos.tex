% DEFINICIÓN DE COMANDOS Y ENTORNOS

% CONJUNTOS DE NÚMEROS

  \newcommand{\N}{\mathbb{N}}     % Naturales
  \newcommand{\Z}{\mathbb{Z}}     % Enteros
  \newcommand{\Q}{\mathbb{Q}}     % Racionales
  \newcommand{\F}{\mathbb{F}}
  \newcommand{\K}{\mathbb{K}}
  %\newcommand{\C}{\mathbb{C}}     % Complejos
  \newcommand{\restoizq}{\text{lrem}}
  \newcommand{\cocienteizq}{\text{lquo}}


% Otros comandos propios
  \newcommand{\mun}{\{\mu_n\}_{n \in \N_0}}                  % Sucesión de momentos
  \newcommand{\C}{\mathcal C}
  \newcommand{\R}{\mathcal R}
  \newcommand{\Ft}{\mathbb{F}(t)[x;\sigma]}
  \newcommand{\intdeg}{\textnormal{intdeg}}
  \newcommand{\extdeg}{\textnormal{extdeg}}






% Para escalar matemáticas:
  \newcommand{\scalemath}[2]{\scalebox{#1}{\mbox{\ensuremath{\displaystyle #2}}}}

% TEOREMAS Y ENTORNOS ASOCIADOS

  % \newtheorem{theorem}{Theorem}[chapter]
  \newtheorem*{teorema*}{Teorema}
  \newtheorem{teorema}{Teorema}[chapter]
  \newtheorem{proposicion}[teorema]{Proposición}
  \newtheorem{lema}[teorema]{Lema}
  \newtheorem{corolario}[teorema]{Corolario}
  \newtheorem{algoritmo}[teorema]{Algoritmo}

    \theoremstyle{definition}
  \newtheorem{definicion}[teorema]{Definición}
  \newtheorem{ejemplo}[teorema]{Ejemplo}

    \theoremstyle{remark}
  \newtheorem{observacion}[teorema]{Observación}
