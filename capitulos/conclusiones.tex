% !TeX root = ../libro.tex
% !TeX encoding = utf8

%\setchapterpreamble[c][0.75\linewidth]{%
%	\sffamily
%  Breve resumen del capítulo. TODO
%	\par\bigskip
%}

\addchap{Conclusiones}\label{ch:Conclusiones}

Este trabajo ha logrado cumplir satisfactoriamente todos los objetivos que fueron planteados en la introducción. 

En el ámbito de las Matemáticas, se han estudiado en profundidad las nociones básicas de los códigos de bloque. También se ha explorado la estructura algebraica de los códigos convolucionales y su ciclicidad mediante los anillos de polinomios sesgados, así como los códigos convolucionales sesgados RS. Además, se ha expuesto en detalle el algoritmo de Sugiyama para los códigos convolucionales sesgados RS, demostrando su eficacia para la corrección de errores. 

En el ámbito de la Ingeniería Informática, se ha implementado en SageMath el algoritmo de Sugiyama para la decodificación de códigos BCH y códigos convolucionales sesgados RS. También se ha implementado un método para la construcción y codificación de estos códigos. Además, hemos probado su eficacia mediante herramientas como PyTest.

Con la realización de este trabajo se han consolidado conocimientos adquiridos a lo largo del doble grado, especialmente en teoría de anillos, teoría de cuerpos y teoría de la información, así como se han incorporado otros nuevos: teoría de módulos, álgebra no conmutativa y teoría de códigos.

Como posible vía futura del trabajo, sería interesante contribuir al proyecto SageMath integrando las implementaciones de los algoritmos de decodificación de Sugiyama y los sistemas de construcción y codificación de códigos convolucionales sesgados RS en el repositorio oficial de SageMath.

En resumen, el Trabajo de Fin de Grado ha sido más que una tarea que requiere tiempo y esfuerzo; ha sido una experiencia académica muy satisfactoria.



\endinput
