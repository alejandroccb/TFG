% !TeX root = ../libro.tex
% !TeX encoding = utf8
%
%*******************************************************
% Summary
%*******************************************************


\chapter*{Resumen}

Los códigos son estructuras algebraicas que permiten transmitir información de manera que el receptor tenga la capacidad de corregir los errores que se produzcan durante la transmisión de un mensaje. En este trabajo, exploraremos los códigos convolucionales cíclicos sesgados, una clase de códigos muy interesante por su estructura algebraica, que facilita el diseño de algoritmos de decodificación eficientes. Además, presentaremos e implementaremos  el algoritmo de Sugiyama para códigos convolucionales cíclicos sesgados utilizando el software SageMath. Para alcanzar estos objetivos, abordaremos varios conceptos fundamentales: comenzaremos con nociones básicas de álgebra, incluyendo teoría de anillos, cuerpos y módulos. Luego, estudiaremos los códigos de bloque, enfocándonos en los códigos lineales y cíclicos. Posteriormente, examinaremos los códigos convolucionales y los anillos de polinomios sesgados, que constituyen la base de los códigos convolucionales cíclicos sesgados. Una vez explicados los códigos convolucionales sesgados, estaremos preparados para estudiar e implementar el algoritmo de Sugiyama para estos códigos.


\paragraph{\textsc{Palabras clave}}
\begin{inparaitem}[\hspace{1em}]
  \item teoría de códigos,
  \item códigos de bloque,
  \item códigos cíclicos,
  \item códigos convolucionales,
  \item SageMath,
  \item algoritmo de Sugiyama.
\end{inparaitem}





\endinput
